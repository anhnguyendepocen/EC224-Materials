\documentclass{article}

\begin{document}

\section{Questions}
\begin{enumerate}
\item What are RKMs? When did they become important in this economic system? 

\item If a tin of jam is worth $\frac{1}{2}$ of margarine, is this a \emph{real} or {nominal} value? 

\item What does the term, ``the cigarette became the standard of value" mean? 

\item How can ``an astute trader`` take advantage of \emph{arbitrage}? 

\item Why do clothing and food prices differ in the shop? 

\item Perfect markets depend on information. When does the free flow of information fall down? 

\item How are cigarettes \emph{clipped}?

\item How does the circulation of hand-rolled cigarettes disrupt the market? 

\item What is \emph{Gresham's Law}?  How does it apply here? 

\item Is there \emph{Money neutrality} in this system? 

\item Why do ``heavy air-raids'' cause deflation? 

\item Why does the relative price of cocoa and soap change in the summer? 

\item Why was there a flight from the BMk?

\item What is the \emph{just price}?

\item Using the diagram of the money market and starting from a point of equilibrium of a price level of 2 and a value of money of $\frac{1}{2}$, show how an expansion of the money supply can increase the price level to 3 and reduce the value of money. 

What is the new value of money?

\item Explain the difference between an increase in inflation and an increase in the cost-of-living

\item What does \emph{classical dichotomy} say about quantitative easing? 

\item How is the real wage defined? 

\item What must hold for $\Delta M = \Delta P$?

\item Use the diagram of the money market to show how worries about the outlook for the stock market and the property market can negate the effect of quantitative easing. 

\item If inflation is expected to be 5\% and the nominal rate of interest for 1 year is 8\%, what is the approximate real interest rate? 

\item What is the real interest rate when the nominal rate is zero and there is 2\% deflation

\item Why should you fear deflation? 

\end{enumerate}

\end{document}
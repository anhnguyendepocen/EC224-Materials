\documentclass[12pt, a4paper, oneside]{article}\usepackage[]{graphicx}\usepackage[]{color}
%% maxwidth is the original width if it is less than linewidth
%% otherwise use linewidth (to make sure the graphics do not exceed the margin)
\makeatletter
\def\maxwidth{ %
  \ifdim\Gin@nat@width>\linewidth
    \linewidth
  \else
    \Gin@nat@width
  \fi
}
\makeatother

\definecolor{fgcolor}{rgb}{0.345, 0.345, 0.345}
\newcommand{\hlnum}[1]{\textcolor[rgb]{0.686,0.059,0.569}{#1}}%
\newcommand{\hlstr}[1]{\textcolor[rgb]{0.192,0.494,0.8}{#1}}%
\newcommand{\hlcom}[1]{\textcolor[rgb]{0.678,0.584,0.686}{\textit{#1}}}%
\newcommand{\hlopt}[1]{\textcolor[rgb]{0,0,0}{#1}}%
\newcommand{\hlstd}[1]{\textcolor[rgb]{0.345,0.345,0.345}{#1}}%
\newcommand{\hlkwa}[1]{\textcolor[rgb]{0.161,0.373,0.58}{\textbf{#1}}}%
\newcommand{\hlkwb}[1]{\textcolor[rgb]{0.69,0.353,0.396}{#1}}%
\newcommand{\hlkwc}[1]{\textcolor[rgb]{0.333,0.667,0.333}{#1}}%
\newcommand{\hlkwd}[1]{\textcolor[rgb]{0.737,0.353,0.396}{\textbf{#1}}}%

\usepackage{framed}
\makeatletter
\newenvironment{kframe}{%
 \def\at@end@of@kframe{}%
 \ifinner\ifhmode%
  \def\at@end@of@kframe{\end{minipage}}%
  \begin{minipage}{\columnwidth}%
 \fi\fi%
 \def\FrameCommand##1{\hskip\@totalleftmargin \hskip-\fboxsep
 \colorbox{shadecolor}{##1}\hskip-\fboxsep
     % There is no \\@totalrightmargin, so:
     \hskip-\linewidth \hskip-\@totalleftmargin \hskip\columnwidth}%
 \MakeFramed {\advance\hsize-\width
   \@totalleftmargin\z@ \linewidth\hsize
   \@setminipage}}%
 {\par\unskip\endMakeFramed%
 \at@end@of@kframe}
\makeatother

\definecolor{shadecolor}{rgb}{.97, .97, .97}
\definecolor{messagecolor}{rgb}{0, 0, 0}
\definecolor{warningcolor}{rgb}{1, 0, 1}
\definecolor{errorcolor}{rgb}{1, 0, 0}
\newenvironment{knitrout}{}{} % an empty environment to be redefined in TeX

\usepackage{alltt} % Paper size, default font size and one-sided paper
%\graphicspath{{./Figures/}} % Specifies the directory where pictures are stored
%\usepackage[dcucite]{harvard}
\author{}
\date{}
\usepackage{rotating}
\usepackage{setspace}
\usepackage{tikz}
\usepackage{pdflscape}
\usepackage{amsmath}
\usepackage[flushleft]{threeparttable}
\usepackage{multirow}
\usepackage[comma, sort&compress]{natbib}% Use the natbib reference package - read up on this to edit the reference style; if you want text (e.g. Smith et al., 2012) for the in-text references (instead of numbers), remove 'numbers' 
\usepackage{graphicx}
%\bibliographystyle{plainnat}
\bibliographystyle{agsm}
\usepackage[colorlinks = true, citecolor = blue, linkcolor = blue]{hyperref}
%\hypersetup{urlcolor=blue, colorlinks=true} % Colors hyperlinks in blue - change to black if annoying
%\renewcommand[\harvardurl]{URL: \url}
\IfFileExists{upquote.sty}{\usepackage{upquote}}{}
\begin{document}
\title{Oligopoly}
\maketitle


\begin{enumerate}
\item For a duopoly with the following demand curve ($Q = 100 - 10P$) and the assumption that the marginal cost of production is zero, 
% latex table generated in R 3.1.1 by xtable 1.7-4 package
% Sun Dec 28 11:09:27 2014
\begin{table}[ht]
\centering
\begin{tabular}{rrr}
  \hline
Q & P & Total.Income \\ 
  \hline
0 & 100 & 0 \\ 
  1 & 90 & 90 \\ 
  2 & 80 & 160 \\ 
  3 & 70 & 210 \\ 
  4 & 60 & 240 \\ 
  5 & 50 & 250 \\ 
  6 & 40 & 240 \\ 
  7 & 30 & 210 \\ 
  8 & 20 & 160 \\ 
  9 & 10 & 90 \\ 
  10 & 0 & 0 \\ 
   \hline
\end{tabular}
\end{table}

\begin{itemize}
\item Explain each firm's profit-maximising position? 
\item Calculate total output, total profit and the profit for each firm? 
\item What is the profit maximising position if there is \emph{collusion}?
\item Explain the term \emph{Nash Equilibrium} identify the Nash Equilibrium for this example,
\end{itemize}

\item How does globalisation affect oligopolistic industries like cars? 

\item Describe the \emph{kinked demand curve}.  What are the consequences of this demand curve for firms? 

\item Looking at figure 16.7 in the textbook, explain why there is no equilibrium at the point where each firm charges 20 pounds for the product.  

\item In the context of Game Theory, give an example of the following
\begin{itemize}
\item \emph{Nash Equilibrium}
\item \emph{Dominant Strategy}
\item \emph{Credible Threat}
\item \emph{Commitment}
\end{itemize}


\item Explain the following terms
\begin{itemize}
\item \emph{Cournot Reaction Function}
\item Residual Demand
\item \emph{First mover advantage}
\item Nash equilibrium for the Bertrand model
\end{itemize}

  
  
\end{enumerate}



\end{document}

\documentclass[14pt,xcolor=pdftex,dvipsnames,table]{beamer}

% Specify theme
\usetheme{Madrid}
% See deic.uab.es/~iblanes/beamer_gallery/index_by_theme.html for other themes
\usepackage{caption}
\usepackage{graphicx}
\usepackage{natbib}
\usepackage{tikz}
\bibliographystyle{agsm}
\graphicspath{{./Figures/}}

% Specify base color
\usecolortheme[named=OliveGreen]{structure}
% See http://goo.gl/p0Phn for other colors

% Specify other colors and options as required
\setbeamercolor{alerted text}{fg=Maroon}
\setbeamertemplate{items}[square]

% Title and author information
\title{An Introduction to Calculus}
\author{Rob Hayward}
\date{}

\begin{document}

\begin{frame}
\titlepage
\end{frame}

%\begin{frame}{Outline}
%\tableofcontents
%\end{frame}

\begin{frame}{Usain Bolt}
%\includegraphics{tikz}
\begin{tikzpicture}[scale = 0.6]
\draw [thick, <->] (0,10) -- (0,0) -- (10,0);
\draw [thin] (0,0) -- (9.58, 10);
\node [below left] at (9,0) {x = seconds};
\node [above, rotate = 90] at (0,9.5) {y = meters};
\draw [<->, blue] (0,0) to [out = 17, in = 195] (9.58, 10.0);
\node [below right] at (9.58, 10.0) {UB};
\end{tikzpicture}
\end{frame}

\begin{frame}{Speed}
\framesubtitle{Average and Instantaneous Speed}
Two lines show average and instantaneous speed
\pause
\begin{itemize}[<+-| alert@+>]
\item $\text{Average Speed} = \frac{Distance}{Time} = \frac{\Delta Distance}{\Delta Time}$
\item $\text{Average Speed} = \frac{100}{9.58} = 10.43\frac{m}{s}$
\item Need to find the tangent to the line for \emph{Instantaneous Speed}
\item $\text{Instantaneous Speed} = \frac{\delta y}{\delta x}$
\end{itemize}
\end{frame}

\begin{frame}{Instantaneous Speed}
\framesubtitle{Small change (h)}
Instantaneous speed is the tangent to the curve.  For $f(x)$
\pause
\begin{itemize}[<+-| alert@+>]
\item $\text{Instantaneous speed} = \frac{f(x_0 + h_0) - f(x_0)}{(x_0 +h_0) - 
x_0}$ 
\item $h_0$ is small
\item $\text{Instantaneous speed} = \frac{f(x_0 + h_0) - f(x_0)}{h_0}$
\end{itemize}
\end{frame}

\begin{frame}{Solution 1}
\framesubtitle{$f(x) = y = x^2$}
\begin{tikzpicture}[yscale = 1/5]
\draw [thick, <->] (0,20) -- (0,0) -- (4,0);
\draw [thick <-] (-4,0) -- (0, 0);
\draw [olive, thick, domain = -4:4] plot (\x, {\x*\x});
\node [below left] at (4,0) {x};
\node [above, rotate = 90] at (,16) {y};
\node [right] at (-4, 16) {$y = x^2$};
\draw (3, 9) circle (0.5pt);
\node [right] at (3, 9) {(3, 9)};
\end{tikzpicture}
\end{frame}

\begin{frame}{Solution 2}
\framesubtitle{Instantaneous change at (3, 9) when $f(x) = x^2$}
$\text{Instantaneous speed} = \frac{f(x_0 + h_0) - f(x_0)}{h_0}$
\vskip 0.25cm
\begin{center}
\rowcolors{1}{OliveGreen!20}{OliveGreen!5}
\begin{tabular}{l r r r}
h & x + h & f(x + h) & $\frac{f(x +h) - f(x)}{h}$\\
\hline
0.1 & 3.1 & 9.61 & 6.1 \\
0.01 & 3.01 & 9.0601 & 6.01 \\
0.001 & 3.001 & 9.0060 & 6.001\\
\end{tabular}
\end{center}
\end{frame}

\begin{frame}{Solution 3}
\framesubtitle{Calculation}
\begin{align*}
\text{Instantaneous speed} = &\frac{f(x_0 + h_0) - f(x_0)}{h_0}\\
 = & \frac{(x_0 + h_0)^2 - x_0^2}{h_0}\\
 = & \frac{x_0^2 +2x_0h_0 + h_0^2 - x_0^2}{h_0}\\
 = & \frac{h_0(2x_0 + h_0)}{h_0}\\
 = & 2x + h
 \end{align*}
 \end{frame}
 
 \begin{frame}{The derivative}
 \framesubtitle{Instantaneous rate of change}
 The instantaneous rate of change 
 \pause
  \begin{itemize}[<+-| alert@+>]
 \item $f'(x_0) = \lim_{h \to 0} \frac{f(x_0 +h) - f(x_0)}{h}$
 \item $f'(x)$
 or
 \item $\frac{\mathrm d y}{\mathrm d x}(x)$
 \end{itemize}
 \pause
 \begin{block}{}
For any positive integer k, the derivative of $f(x) = x^k$ at $x_0$ is $f'(x) = kx^{k-1}$
\end{block}
\end{frame}

\begin{frame}{Example 1}
\begin{block}{Derivative and the marginal}
Marginal is \emph{a very small change} - like a derivative
\end{block}
\pause
\begin{block}{Total revenue and total cost}
\begin{align*}
TR &= 201 -2Q + Q^2\\
MR &= -2 + 2Q\\
TC &= 20Q -2Q^2\\
MC &= 20 -4Q
\end{align*}
\end{block}
\end{frame}

\begin{frame}{MR = MC}
For profit maximisation, 
\begin{block}{MR = MC}
\begin{align*}
20 - 4Q &= 2Q - 2\\
22 &= 6Q\\
Q &= 3.67
\end{align*}
\end{block}
\end{frame}

\begin{frame}{Maximum and minimum}
\begin{block}{}
The slope of the curve will be equal to zero at a maximum or minimum
\end{block}
\pause
This can be used to find the maximum or minimum
\pause
\begin{itemize}[<+-| alert@+>]
\item Find the derivative
\item Set it equal to zero
\item Solve
\end{itemize}
\end{frame}

\begin{frame}{Example 3}
\begin{block}{Utility function}
\begin{equation*}
U = 5X - 0.2X^2
\end{equation*}
\end{block}
\begin{block}{Find derivative}
\begin{equation*}
\frac{\delta U}{\delta X} = 5 - 0.4X
\end{equation*}
\end{block}
\end{frame}

\begin{frame}{Finding maximum}
The slope must be zero at the maximum
\pause
\begin{block}{}
\begin{align*}
\frac{\delta U}{\delta X} &= 5 - 0.4X\\
0 &= 5 - 0.4X\\
0.4X &= 5\\
X = 12.5
\end{align*}
\end{block}
\end{frame}

\begin{frame}{Example 4}
\framesubtitle{$TPP  = 100 + 32L + 10L^2 - L^3$}
\begin{tikzpicture}[yscale = 1/50, scale = 0.6]
\draw [thick, <->] (0,500) -- (0,0) -- (10,0);
\draw [olive, thick, domain = 0:10] plot (\x, {100+32*\x+10*\x*\x-\x*\x*\x});
\node [right] at (10, 420) {TPP};
\end{tikzpicture}
\end{frame}

\begin{frame}{Example 2 page 2}
{Differentiating TPP}
\begin{align*}
TPP  =& 100 + 32L + 10L2 - L^3\\
TPP' =& 32 +20L -3L^2
\end{align*}
Gradient at maximum is zero, therefore
\begin{align*}
3L^2 - 20L -32 = & 0\\
(3L + 4)(L - 8) =& 0
\end{align*}
So $L = 8$, or $L = -1.33$ 
\end{frame}

\begin{frame}{Quadratic Solution}
For a quadratic equation of the form
\begin{equation*}
ax^2 +bx + c
\end{equation*}

The \emph{solution} or the roots can be found with 
\begin{block}{}
\begin{equation*}
x = \frac{-b \pm \sqrt{b^2 - 4ac}}{2a}
\end{equation*}
\end{block}
\end{frame}
\end{document}
\documentclass[12pt, a4paper, oneside]{article}\usepackage[]{graphicx}\usepackage[]{color}
%% maxwidth is the original width if it is less than linewidth
%% otherwise use linewidth (to make sure the graphics do not exceed the margin)
\makeatletter
\def\maxwidth{ %
  \ifdim\Gin@nat@width>\linewidth
    \linewidth
  \else
    \Gin@nat@width
  \fi
}
\makeatother

\definecolor{fgcolor}{rgb}{0.345, 0.345, 0.345}
\newcommand{\hlnum}[1]{\textcolor[rgb]{0.686,0.059,0.569}{#1}}%
\newcommand{\hlstr}[1]{\textcolor[rgb]{0.192,0.494,0.8}{#1}}%
\newcommand{\hlcom}[1]{\textcolor[rgb]{0.678,0.584,0.686}{\textit{#1}}}%
\newcommand{\hlopt}[1]{\textcolor[rgb]{0,0,0}{#1}}%
\newcommand{\hlstd}[1]{\textcolor[rgb]{0.345,0.345,0.345}{#1}}%
\newcommand{\hlkwa}[1]{\textcolor[rgb]{0.161,0.373,0.58}{\textbf{#1}}}%
\newcommand{\hlkwb}[1]{\textcolor[rgb]{0.69,0.353,0.396}{#1}}%
\newcommand{\hlkwc}[1]{\textcolor[rgb]{0.333,0.667,0.333}{#1}}%
\newcommand{\hlkwd}[1]{\textcolor[rgb]{0.737,0.353,0.396}{\textbf{#1}}}%

\usepackage{framed}
\makeatletter
\newenvironment{kframe}{%
 \def\at@end@of@kframe{}%
 \ifinner\ifhmode%
  \def\at@end@of@kframe{\end{minipage}}%
  \begin{minipage}{\columnwidth}%
 \fi\fi%
 \def\FrameCommand##1{\hskip\@totalleftmargin \hskip-\fboxsep
 \colorbox{shadecolor}{##1}\hskip-\fboxsep
     % There is no \\@totalrightmargin, so:
     \hskip-\linewidth \hskip-\@totalleftmargin \hskip\columnwidth}%
 \MakeFramed {\advance\hsize-\width
   \@totalleftmargin\z@ \linewidth\hsize
   \@setminipage}}%
 {\par\unskip\endMakeFramed%
 \at@end@of@kframe}
\makeatother

\definecolor{shadecolor}{rgb}{.97, .97, .97}
\definecolor{messagecolor}{rgb}{0, 0, 0}
\definecolor{warningcolor}{rgb}{1, 0, 1}
\definecolor{errorcolor}{rgb}{1, 0, 0}
\newenvironment{knitrout}{}{} % an empty environment to be redefined in TeX

\usepackage{alltt} % Paper size, default font size and one-sided paper
%\graphicspath{{./Figures/}} % Specifies the directory where pictures are stored
%\usepackage[dcucite]{harvard}
\usepackage{rotating}
\usepackage{setspace}
\usepackage{pdflscape}
\usepackage[flushleft]{threeparttable}
\usepackage{multirow}
\usepackage{amsmath}
\usepackage[comma, sort&compress]{natbib}% Use the natbib reference package - read up on this to edit the reference style; if you want text (e.g. Smith et al., 2012) for the in-text references (instead of numbers), remove 'numbers' 
\usepackage{graphicx}
%\bibliographystyle{plainnat}
\bibliographystyle{agsm}
\usepackage[colorlinks = true, citecolor = blue, linkcolor = blue]{hyperref}
%\hypersetup{urlcolor=blue, colorlinks=true} % Colors hyperlinks in blue - change to black if annoying
%\renewcommand[\harvardurl]{URL: \url}
\IfFileExists{upquote.sty}{\usepackage{upquote}}{}
\begin{document}
\title{Questions on Calculus}
\author{Rob Hayward} 
\date{\today}
\maketitle

\doublespacing
\begin{enumerate}


\item What are the drivatives of
\begin{itemize}
\item $y = 2 + 6x$

$\frac{d(y)}{d(x)} = 6$

\item $y = 5 - 4x +2x^3$

$\frac{d(y)}{d(x)} = 4 + 6x$

\item $y = 25 +6x^2 - 3x^3 +25x^4$

$\frac{d(y)}{d(x)} = 12x - 9x^2 + 100x^3$

\item $y - 3 = 2x$

$\frac{d(y)}{d(x)} = 2$

\item $TPP = 24 +5L +2L^2 - L^3$

$\frac{d(TPP)}{d(L)} = 5 +4L - 3L^2$

\item What does your answer to the prevous question tell you about the shape of the Total Physical Product Curve? 

It goes up initially but starts rise at a slower pace.  There are diminishing returns.  

Find the second derivative of the following

\item $TU = 25 + X_1 - X_1^2$

$\frac{d^2(U)}{d(X_1)} = -2$

\item $TU = 25 +25X_1 -2X_1^2$

$\frac{d^2(U)}{d(X_1)} = - 4$

\item $TPP = 15 +15Q +Q^2 - Q^3$

$\frac{d^2(TPP)}{d(L)} = 2 -6Q$

\item What does your answer to the previous question tell you about the shape of the Total Physical Product Curve? 

It shows that the rate of change is falling.   

\item Here is the total physical product curve. Show the \emph{average physical product}? for a particular point on the graph. 

  
\begin{knitrout}
\definecolor{shadecolor}{rgb}{0.969, 0.969, 0.969}\color{fgcolor}

{\centering \includegraphics[width=\maxwidth]{figure/plot-1} 

}



\end{knitrout}
\item How would you calculate the \emph{marginal physical product}?

\end{itemize}

\item What is the gradient of the TPP at its peak? 

\item What is the value of the MPP when TPP is at its peak? 

\item Given the $TPP = 100 + 32Q +23Q^2 - Q^3$, 
\begin{itemize}
\item What is the TPP' or MPP?

$32 + 46Q - 3Q^2$


\item How would you find the maximum TPP?

\begin{align*}
3Q^2 - 46Q -32 &= 0\\
(3Q + 2)(Q - 16) &= 0
\end{align*}
Therefore
$3Q = -2 \quad \text{or} \quad Q = 16$
\end{itemize}

\item Given the $TPP = 500 + 180Q + 15Q^2 - 2Q^3$, 
\begin{itemize}
\item What is the TPP' or MPP?

$MPP = 180 + 30Q -6Q^2$

\item How would you find the maximum TPP?

\begin{align*}
6Q^2 - 30Q - 180 &= 0\\
(6Q + 90)(Q - 20) &= 0
\end{align*}
So
$Q = -15 \quad \text{or} \quad Q = 20$. 

\end{itemize}


\item Given the $TU = 25X - 0.5X^2$
\begin{itemize}
\item What is the drivative of TU?

$TU = 25 - X$

\item What can we say about the utility of X?

There are diminishing returns. 

\end{itemize}

Given a demand curve 
\begin{equation*}
Q_d = 90 - 5P - 0.2P^2
\end{equation*}

\begin{knitrout}
\definecolor{shadecolor}{rgb}{0.969, 0.969, 0.969}\color{fgcolor}
\includegraphics[width=\maxwidth]{figure/plot3-1} 

\end{knitrout}

What is the elasticity of demand at the point $P = 5, Q = 70$?

The equation for elasticity is 
\begin{align*}
\varepsilon_d =& \frac{d(Q)}{d(P)}\times \frac{P}{Q}\\
               =& \frac{-5 - 0.4 \times 5}\times \frac{5}{70}\\
              =& -5 -2 \times \frac{5}{70}\\
              = -0.5
\end{align*}

Therefore the demand is inelastic

\end{enumerate}
\end{document}
